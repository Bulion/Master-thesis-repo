\documentclass[printmode,en]{mgr}
%opcje klasy dokumentu mgr.cls zostały opisane w dołączonej instrukcji

%poniżej deklaracje użycia pakietów, usunąć to co jest niepotrzebne
%\usepackage{polski}       %przydatne podczas składania dokumentów w
%j. polskim
\usepackage[english]{babel} %alternatywnie do pakietu
%polski, wybrać jeden z nich
\usepackage[utf8]{inputenc} %kodowanie znaków, zależne od systemu
\usepackage[T1]{fontenc} %poprawne składanie polskich czcionek

\usepackage{hyperref}

%pakiety do grafiki
\usepackage{graphicx}
\usepackage{psfrag}
\usepackage{subcaption}

\graphicspath{{../graphics/}}

%pakiety dodające dużo dodatkowych poleceń matematycznych
\usepackage{amsmath}
\usepackage{amsfonts}

%pakiety wspomagające i poprawiające składanie tabel
\usepackage{supertabular}
\usepackage{array}
\usepackage{tabularx}
\usepackage{hhline}

%pakiet wypisujący na marginesie etykiety równań i rysunków
%zdefiniowanych przez \label{}, chcąc wygenerować finalną wersję
%dokumentu wystarczy usunąć poniższą linię
%\usepackage{showlabels}

%definicje własnych poleceń
\newcommand{\R}{I\!\!R} %symbol liczb rzeczywistych, działa tylko w
                        %trybie matematycznym
\newtheorem{theorem}{Twierdzenie}[section] %nowe otoczenie do
                                           %składania twierdzeń

%dane do złożenia strony tytułowej
\title{Zastosowanie okularów rozszerzonej rzeczywistości w aplikacjach
robotycznych}
\engtitle{Augmented reality goggles in robotic applications}
\author{Dawid Śliwa}
\supervisor{dr inż. Janusz Jakubiak, I-6}
%\guardian{dr hab. inż. Imię Nazwisko Prof. PWr, I-6} %nie używać
%jeśli opiekun jest tą samą osobą co prowadzący pracę

%\date{2008} %standardowo u dołu strony tytułowej umieszczany jest
%bieżący rok, to polecenie pozwala wstawić dowolny rok

%poniżej jest lista kierunków i specjalności na wydziale elektroniki,
%należy wybrać właściwe lub dopisać jeśli nie ma odpowiednich
\field{Control Engineering and Robotics (AIR)}
\specialisation{Embedded Robotics (AER)}
%\specialisation{Komputerowe sieci sterowania (ARK)}
%\specialisation{Systemy informatyczne w automatyce (ASI)}
%\specialisation{Komputerowe systemy zarządzania \\procesami
%produkcyjnymi (ARS)} \field{Elektronika i telekomunikacja (EIT)}
%\specialisation{Akustyka (ETA)} \specialisation{Aparatura
%elektroniczna (EAE)} \specialisation{Elektroniczne i komputerowe
%\\systemy automatyki (ESA)} \specialisation{Zastosowania inżynierii
%komputerowej \\w technice (EZI)} \specialisation{Inżynieria dźwięku
%(EID)} \specialisation{Elektronika stosowana \\i optokomunikacja
%(TEO)} \specialisation{Telekomunikacyjne sieci szerokopasmowe (TSS)}
%\specialisation{Teleinformatyczne sieci mobilne (TSM)}
%\specialisation{Sygnały w telekomunikacji cyfrowej (TSC)}
%\specialisation{Teleinformatyczne systemy rozsiewcze (TSR)}
%\field{Informatyka (INF)} \specialisation{Systemy informatyki w
%medycynie \\i technice (IMT)} \specialisation{Inżynieria systemów
%informatycznych (INS)} \specialisation{Inżynieria internetowa (INT)}
%\specialisation{Systemy i sieci komputerowe (ISK)}
%\field{Teleinformatyka (TIN)} \specialisation{Teleinformatyka (TIN)}

%tutaj zaczyna się właściwa treść dokumentu
\begin{document}
\bibliographystyle{plabbrv} %tylko gdy używamy BibTeXa, ustawia polski
                            %styl bibliografii

\maketitle %polecenie generujące stronę tytułową

%\dedication{6cm}{To jest przykładowa treść opcjonalnej dedykacji,
%  należy ją zmienić lub usunąć w całości polecenie
%  \texttt{$\backslash$dedication}}

\tableofcontents %spis treści

\chapter{Introduction}
Robots become more an more often seen in our environment. Starting from nowadays standard industrial applications and ending on home appliances robots. They all have more or less user friendly interface created to programm or control them. In factories can be seen most often stationary or handheld controllers and in consumer appliances, smartphone almost every time is used. Problem is that, this kind of interaction is not natural for humans. For comparison, communication between two employes working together is mostly done by voice, gestures and sometime touch. That is why modern controllers should been using these. This could improve a way of interaction on human-machine level. \\

A few years ago, a revolution called Industry 4.0 began which most important statement was to not replace peoples in factories by machines, but allow them to cooperate at production line. From that time companies are trying to simplyfy teaching process of robots and give them ability to sense the changing environment. Also enhancements are done on the other side. Employees are equipped with many solutions with are extending they perception. This is allowing to get better undersatnding what machines are doing or even see what they are "thinking".

\section{Purpose and scope of work}
This thesis will focus on Augmented Reality and they ussage in modern factories and research facilities. At the beginning, different types of AR technologies will be compared to give overall view on how this is working. Then industrial or commercial products which are available right now on the market will be presented. The last part of topic studies will try to present selected solutions with are already used in real world applications.\\

Research part of this thesis will try to present simple examples of implementation AR in robotic applications. The topic will cover the issue of planing movement of robotic arm and also controlling and presenting data from mobile robot. This should give more or less understanding what this technology is capable and whats are its current limitations.


\chapter{Introduction to Augmented Reality}
%tutaj można napisac cos o tym gdzie można umiejscowic w przestrzeni augmented reality (relity<->mixed reality<->virtual reality)
The perception of our surroundings is made to a large extent by the organ of sight. Thanks to that we are able to navigate and operate in our real environment. But what if we will try to trick him by placing displays in front of our eyes? Depending on content generated by computer it could simply show some additional information or create ilusion of being completely somewhere else. To distinguish types of immersion the concept of a "Reality-Virtuality Continuum" was created. It graphical representation is shown on figure \ref{fig:RVContinuum}.

\begin{figure}[!ht]
  \centering
    \includegraphics[width=0.8\textwidth]{RVContinuum}
  \caption{Reality-Virtuality (RV) Continuum}
  \label{fig:RVContinuum}
\end{figure}

On the most left side of this line there is our real environment with real objects in no way disturbed by computer graphics. On the other hand on the most right side there is Virtual Reality with fully 3D generated world that could even in example not holding known by us laws of physics. Between two of those is everything with is mixing one part with another. Depending on what balance is, then we will talk about Augmented Reality or Augmented Virtuality. For example when there operator is getting some simulated cues to augment his natural feedback then it is AR. When in virtual world appears some real life objects or persons then it is called AV. Good example for that are modern news where reporters are working at green screen and in television they appear in 3D generated studios. As this paper will only discuss topic of using Augmented Reality in robotics applications that is why technical aspects of Augmented Virtuality or Virtual Reality will be not considered.

\section{Technology overview}
AR devices can take many different forms but way of processing data is almost always the same. In the simplest way it could be explained as following process. First device need to capture an image and localize itself or the user in environment then mix those data with CG objects and at the end display them on a screen. Figure \ref{fig:ARpipeline} presents this pipeline in graphical form with few additional steps.

\begin{figure}[!ht]
  \centering
    \includegraphics[width=0.8\textwidth]{ARPipeline}
  \caption{Simplified AR pipeline}
  \label{fig:ARpipeline}
\end{figure}

This section will show different types of technologies with where used at implementation of individual elements of this process.

\subsection{Types of image projection}
%wymienic typy projekcji obrazu
There is several way of displaying virtual objects in our environment. Depending on its type they could be more or less immersive for the user.\\

The most popular and the simplest types of projection is video-mixing. It involves the usage of devices equipped with a camera and standard display to present AR content. In most cases smartphones or tablets are used because they have built-in every needed component and they are very portable. Figure \ref{fig:tabletAR} present example of such projection. The great advantage of this approach compared to others is that you can use one device in cooperation with other people, which could result in a significant reduction of operating costs. The disadvantages should be mentioned that the employee is not able to observe the environment and use both hands to do his job. Therefore, this type of projection is most often used in devices used to supervise the operation of machines.

\begin{figure}[!ht]
  \centering
    \includegraphics[width=0.8\textwidth]{tablet_AR}
  \caption{Worker using tablet as video-mixing device}
  \label{fig:tabletAR}
\end{figure}

Second type of projection is slightly different than first one. It is called spatial display. In this case virtual objects are shown directly on real environment surfaces by usage of digital projectors. It could be realised on two way. We could have handheld device with will work similar to flashlight or stationary mounted projector. In first case we could for example inspect virtual paths of transportation robots in warehouses or factory by highlighting floors. Also autonomous cars can use their headlights to display information on the road and for instance inform pedestrian that they could safely cross the passage. Stationary projectors are most often used with collaborative robots. They can display for example some cues for worker which item it will pick-up or where it will place it on shared workspace. Presented types of projection could be seen on Figure \ref{fig:spatialAR}.

\begin{figure}[!ht]
\centering
\begin{subfigure}{.4\textwidth}
  \centering
  \includegraphics[width=.9\linewidth]{augmentedprojectors}
  \caption{Augmented projector}
  \label{fig:augmentedprojectors}
\end{subfigure}%
\begin{subfigure}{.4\textwidth}
  \centering
  \includegraphics[width=.8\linewidth]{spatialARcowork}
  \caption{Man and robot coworking}
  \label{fig:spatialARcowork}
\end{subfigure}
\caption{Examples of spatial projection}
\label{fig:spatialAR}
\end{figure}

Next AR display technology which will be presented is optical see-through. Its principle of operation boils down to projecting images on partially reflecting surface. This allow to combine real world view with generated graphics which will appear as floating in space holograms. There are two ways to achieve this goal. Device could use a head-mounted or head-up displays. Figure \ref{fig:seeThroughAR} shows how the projection is carried out with particular case. First type presented at image \ref{fig:headMounted} is commonly used in situations where worker need to have both hands free to do his work. At HMD could be shown some step by step instructions or in case operating with robots, live sensors data. Example from image \ref{fig:headUp} show the most common scenario of usage of HUD. Currently they are showing informations about speed and navigation guidance for a driver but in autonomous vehicles they display also some cues what car see and what he intends to do.

\begin{figure}[!ht]
\centering
\begin{subfigure}{.5\textwidth}
  \centering
  \includegraphics[width=.8\linewidth]{HMD}
  \caption{Head-Mounted Display}
  \label{fig:headMounted}
\end{subfigure}%
\begin{subfigure}{.5\textwidth}
  \centering
  \includegraphics[width=\linewidth]{HUD}
  \caption{Head-Up Display}
  \label{fig:headUp}
\end{subfigure}
\caption{Examples of optical see-through projections}
\label{fig:seeThroughAR}
\end{figure}

\subsection{Positioning and location}
%rodzaje lokalizacji obiektow w scenie i pozycjonowania samych gogli w przestrzeni
Only small part of AR devices is presenting data that is not depending on real environment obstacles. These include basic versions of HUD and smartglasses. Rest of them need somehow localize itself or detect position of the user. In this section various types of positioning techniques will be presented.\\

Marker-based localization is simplest one. The principle of operation consists in detection of predefined shape or image by the device camera. To achieve this first of all data need to be preprocesed.

Most trivial example is when our marker have square shape. In that case algorithm need only know what is length of side. With this information it is able to calculate position and orientation of the device basing on detected corrners. As the most important are only the edges the center can have any shape. This allow to store some information inside and also help to determinate from with side of rectangle camera is looking. Figure \ref{fig:ARMarker} show detected marker with read associated to it ID number. Pros of this approach is quite low requirement for computational power and easy way to generate many unique ID's. The disadvantage is that the user must print these images and arrange them in their chosen locations.

There is also posibility to track normal images or even 3D objects. In this case there are used algorithms with are extracting some particular features from provided data (Figure \ref{fig:featureExtraction}). Basing on them device could be able to localize itself in space only by looking on for example machine logo. Pros of this approach is better tracking of object even when whole image is not in camera view. Disadvantage is more complex computation.

\begin{figure}[!ht]
\centering
\begin{subfigure}{.5\textwidth}
  \centering
  \includegraphics[width=.8\linewidth]{ARMarker}
  \caption{Detected ArUco marker with ID}
  \label{fig:ARMarker}
\end{subfigure}%
\begin{subfigure}{.5\textwidth}
  \centering
  \includegraphics[width=.65\linewidth]{featureExtraction}
  \caption{Image with marked extracted features}
  \label{fig:featureExtraction}
\end{subfigure}
\caption{Example of markers used for localization purposes}
\label{fig:markerBasedLocalization}
\end{figure}

Next discussed type of localization is based on measurements obtained via external devices. There are many possible implementations of this approach but they could be categorised based upon the working principle.

Optical tracking of passive markers is best known in film industry. It is used in Motion Capture studios to transfer movement of actor to 3D modeled characters. It requires usage of multiple high-speed cameras with infrared illuminators fixed around the measurment area to traingulate a reflective marker position. Successful capture of tracked point by at least two camera give sub-milimeter precision. To avoid situation where marker is occluded by some obstacles localized object could be equipped with redundant ones. Disadvantage of such system is limitation of operating area caused by strength of the reflected light. There is also a way to increase to a certain extent range by using active markers which are light sources but this cause another problems with need to provide power to those.

From optical localization method there are also systems working without any markers. They are based on 3D depth cameras with are providing not only image but also give information about distance from objects. This allow to not only get position of tracked device but also could give feedback about scanned environment. This kind of systems generally have operation distance from 0.5 to 8 meters. Precision of positioning it is inversely proportional to it.

External localization could be also realized by measuring field strength or time of flight of electromagnetic waves. In this case at least three transmiters are needed to estabilish position. This method is slighty less accurate in compare to optical ones but have huge advantage in the form of beeing able to track device without direct visibility. Also range and refresh rate of such systems is significantly larger than optical ones.

\begin{figure}[!ht]
\centering
\begin{subfigure}{.33\textwidth}
  \centering
  \includegraphics[width=.8\linewidth]{ARMarker}
  \caption{Motion capture studio}
  \label{fig:1}
\end{subfigure}%
\begin{subfigure}{.33\textwidth}
  \centering
  \includegraphics[width=.65\linewidth]{featureExtraction}
  \caption{3D Camera Depth Data}
  \label{fig:2}
\end{subfigure}
\begin{subfigure}{.33\textwidth}
  \centering
  \includegraphics[width=.65\linewidth]{featureExtraction}
  \caption{UWB Localization module}
  \label{fig:3}
\end{subfigure}
\caption{Example of markers used for localization purposes}
\label{fig:markerBasedLocalization}
\end{figure}

Last presented type of positioning is Simultaneous Localization And Mapping (SLAM). This is technique with is using Laser Range Sensors or 3D Depth cameras to create map of environment. Basing on that data algorithm could calculate position and orientation of device in space.


\subsection{User control and interaction}
%sposoby interakcji uzytkownikow z wirtualnymi obiektami (gesty, głos, kontrolery)

\section{Products available on market}

\subsection{Industrial grade}

\subsection{Consumer appliances}

\subsection{Comparison}

\chapter{Applications of Augmented Reality}


\chapter{Research of the subject}

\section{Used technologies}

\subsection{Unity}

\subsection{Vuforia}

\subsection{Robotic Operating System}

\section{Test results}

\subsection{Robotic arm}

\subsection{Mobile robot}


\chapter{Summary}
\appendix


\addcontentsline{toc}{chapter}{References} %utworzenie w
                                             %spisietreści pozycji
                                             %Bibliografia

\bibliography{References} % wstawia bibliografię korzystając z pliku
                            % bibliografia.bib - dotyczy BibTeXa,
                            % jeżeli nie korzystamy z BibTeXa należy
                            % użyć otoczenia thebibliography

%opcjonalnie może się tu pojawić spis rysunków i tabel
% \listoffigures
% \listoftables
\end{document}
